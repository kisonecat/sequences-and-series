\nobreak The harmonic series diverges, but so does the series
\[
\sum_{n=100}^\infty \frac{1}{n} = \frac{1}{100} + \frac{1}{101} + \frac{1}{102} + \frac{1}{103} + \cdots
\]
Convergence doesn't depend on the beginning of the series; whether or not I include the first 99 terms
\[
\frac{1}{1} + \frac{1}{2} + \frac{1}{3} + \cdots + \frac{1}{97} + \frac{1}{98} + \frac{1}{99}
\]
does affect whether or the sum of \textit{all} the terms diverges.  In
short, convergence depends not on how a series begins, but on how a
series ends.  The end of a series is sometimes called the
\defnword{tail} of the series.

\begin{definition}
  Let $N > 1$ be an integer, and consider a series $\sum_{n=1}^\infty a_n$.  The series we get by removing the first $N-1$ terms, namely
\[
\sum_{n=N}^\infty a_n
\]
is called a \defnword{tail} of the given series.
\end{definition}

Here is the theorem that describes how tails relate to convergence.
\begin{theorem}\label{thm:convergence-for-tails}
  Let $N > 1$ be an integer.  The series $\sum_{n=1}^\infty a_n$ converges if and only if $\sum_{n=N}^\infty a_n$ converges.
\end{theorem}
This could be shortened to ``The series converges iff a tail of the
series converges,'' or even just to the slogan that convergence
depends on the tail.
\begin{proof}
  Suppose $\sum_{n=1}^\infty a_n$ converges to $L$, meaning
\[
\lim_{M \to \infty} \sum_{n=1}^M a_n = L.
\]
In that case, applying limit laws reveals
\begin{align*}
\lim_{M \to \infty} \sum_{n=N}^M a_n
&= \lim_{M \to \infty} \left( \left( \sum_{n=1}^M a_n \right) - \left( \sum_{n=1}^{N-1} a_n \right) \right) \\
&= \lim_{M \to \infty} \left( \sum_{n=1}^M a_n \right) - \lim_{M \to \infty} \left( \sum_{n=1}^{N-1} a_n \right) \\
&= \lim_{M \to \infty} \left( \sum_{n=1}^M a_n \right) - \left( \sum_{n=1}^{N-1} a_n \right) \\
&= L - \left( \sum_{n=1}^{N-1} a_n \right), \\
\end{align*}
which means the series $\sum_{n=N}^M a_n$ converges.

The other direction is left to you, the reader.
\end{proof}

\begin{example}
Does the series $\displaystyle\sum_{n=153}^\infty \frac{1}{n^2}$ converge?
\end{example}
\begin{solution}
  Yes! This series is a tail of the convergent $p$-series $\displaystyle\sum_{n=1}^\infty \frac{1}{n^2}$; in this case, $p = 2$.
\end{solution}

You might recall Example~\xrefn{example:n-to-fifth-over-five-to-n}, which we'll redo here.
\begin{example}
  Show that $\ds\sum_{n=0}^\infty {n^5\over 5^n}$ converges by using the comparison test and Theorem~\xrefn{thm:convergence-for-tails}.
\end{example}

\begin{solution}
  The given series converges iff $\ds\sum_{n=23}^\infty {n^5\over 5^n}$ by Theorem~\xrefn{thm:convergence-for-tails}.  But whenever $n \geq 23$, we have
  $$
  \frac{n^5}{5^n} \leq \frac{2^n}{5^n} = \left( \frac{2}{5} \right)^n
  $$
  The series $\ds\sum_{n=23}^\infty \left( \frac{2}{5} \right)^n$
  converges, since it is a geometric series with common ratio $2/5$,
  so by comparison, the smaller series $\ds\sum_{n=23}^\infty
  {n^5\over 5^n}$ also converges.
\end{solution}

In light of Theorem~\xrefn{thm:convergence-for-tails}, many textbooks
will choose to write
\[
\sum_n a_n
\]
instead of $\displaystyle\sum_{n=1}^\infty a_n$ when discussing
convergence.  Whether or not the series converges doesn't depend on
the initial index, so if we want to state theorems about convergence,
we can avoid potentially distracting details by simply not speaking
about where the series begins.
